\documentclass{article}
\title{Guidelines for the C programming project}
\author{Paolo Bonzini, Adina Mosincat}
\date{\today}
\def\UL{\char`\_}

\begin{document}

\maketitle

\section{Game parameters}

\subsection{Battlefield}

        The battlefield is a 1,000 by 1,000 meter square.  A wall
        surrounds the perimeter, so that a robot running into the wall
        will incur damage.

        The lower left corner has the coordinates x = 0, y = 0; the upper
        right corner has the coordinated x = 999, y = 999.

        The compass system is oriented so that east (right) is 0
        degrees, 90 is north, 180 is west, 270 is south.  One degree
        below east is 359.


\begin{verbatim}
     135    90   45
         \  |  / 
          \ | /
    180 --- x --- 0
          / | \
         /  |  \ 
     225   270   315
\end{verbatim}


\subsection{Robot offense}

        The main offensive weapons are the cannon and scanner.  The
        cannon has a range of 700 meters.  There are an unlimited number
        of missiles that can be fired, but a reloading factor limits the
        number of missiles in the air at any one time to two.  The cannon
        is mounted on an independent turret, and therefore can fire any
        direction, 0-359, regardless of robot heading.

        The scanner is an optical device that can instantly scan any
        chosen heading, 0-359.  The scanner has a maximum resolution of  
        +/- 10 degrees.  This enables the robot to quickly scan the field
        at a low resolution, then use maximum resolution to pinpoint an
        opponent.


\subsection{Robot defense }

        The only defense available are the robot engine and status
        registers.  The motor can be engaged on any heading, 0-359, in
        speeds from 0-100 percent of power.  There are acceleration and
        deacceleration factors.  A speed of 0 stops the motor.  Turns can
        be negotiated at speeds of 50% and less, in any direction.  Of
        course, the engine can be engaged any time, and is necessary
        on offense when a target is beyond the 700 meter range of the
        cannon.

        Certain status registers provide feedback to the robot. The
        primary registers indicate the percent of damage, and current x
        and y locations on the battlefield.  Another register provides
        current movement speed.


\subsection{Disabling opponents}

        A robot is considered dead when the damage reaches 100\%.  The
        amount of damage inflicted during the game is as follows:

\begin{description}
\item[2\%] upon collision into another robot (both robots in a    
                     collision receive damage) or into a wall.  A
                     collision also causes the robot engine to disengage,
                     and speed is reduced to 0.

\item[3\%] when a missile explodes within a 20--40 meter radius.

\item[5\%] when a missile explodes within a 5--20 meter radius.

\item[10\%] when a missile explodes within a 0--5 meter radius.
\end{description}

        Damage is cumulative, and cannot be repaired.  However, a robot
        does not lose any mobility, fire potential, etc. at high damage
        levels.  In other words, a robot at 99\% damage performs equally
        as a robot with no damage.

        A robot program that crashes or exits (!) will bring immediately
        the robot to 100\% damage.

\section{Simulation execution}

        The server is started and given on the command line the number of
        robots participating in the simulation.  When the given number of
        clients is connected, the simulation starts.

        The simulation proceeds in cycles.  The sequence of operations in
        a cycle is as follows:
        \begin{enumerate}
        \item Send current location to the robots
        \item Move and explode missiles
        \item Update display
        \item Wait 10 ms (minus the time that was used to update the display)
        \item Receive commands from the robots
        \item Send command replies to the robots
        \item Move robots
        \end{enumerate}

        Note that the robots will not necessarily interact with the server
        on every cycle.  Therefore, it is okay to modify the above protocol
        so that communication between server and client only happens
        when there is a need for it.

        After a number of cycles given on the command line to the server
        the server sends a command that will cause the robots to exit.

\section{Robots Function Library}

        The function library provides control of the robot.  The library
        is included from \texttt{robots.h}.  The first time a function from
        the library is called, the library will attempt to connect to a server
        (which must be already running).

\begin{description}
\item[\texttt{int scan (int degree,int resolution)}]
        invokes the robot's scanner, at a specified
        angle and resolution.  It returns 0 if no robots are
        within the scan range or a positive integer representing the
        range to the closest robot.  Angles not between 0 and 359 are accepted.
        Resolution controls the scanner's sensing resolution, up to
        $\pm 10$ degrees.

        Examples:
\begin{verbatim}
range = scan(45,0);   /* scan 45, with no variance */
range = scan(365,10); /* scans the range from 355 to 15 */
\end{verbatim}

\item[\texttt{int cannon (int degree,int range)}]
        fires a missile heading a specified range
        and direction.  It returns 1 (true) if a missile was fired,
        or 0 (false) if the cannon is reloading.  ``Out of range'' angles
        are accepted as in scan().  Range can be 0-700, with greater
        ranges truncated to 700.

        Example:
\begin{verbatim}
degree = 45;                    /* direction to test... */
if ((range=scan(degree,2)) > 0) /* see if a target is there */
  cannon(degree,range);         /* if so, fire a missile */
\end{verbatim}

\item[\texttt{int drive (int degree,int speed)}]
        activates the engine, on a 
        specified heading and target speed.  Degree is forced into the range
        0-359 as in scan().  Speed is expressed as a percent, with 100 as
        maximum.  A speed of 0 disengages the drive.  Changes in
        direction can be made only at speeds of less than 50 percent,
        otherwise the degree argument will be silently ignored (and \emph{only}
        the degree argument).

        The exact ``unit of measure'' of the speed argument will be
        decided during the project, and so will the precise semantics of
        acceleration and deceleration.  However, decelerating from speed
        20 to speed 0 should take exactly 8 meters.

        Example:
\begin{verbatim}
drive(0,100);  /* head due east, at maximum speed */
drive(90,0);   /* stop motion */
\end{verbatim}

\item[\texttt{int damage()}]
        The damage() function returns the current amount of damage
        incurred.  damage() takes no arguments, and returns the percent
        of damage, 0-99. (100 percent damage means the robot is
        completely disabled, thus no longer running!) 

        Example:
\begin{verbatim}
d = damage();       /* save current state */
...                 /* other instructions */
if (d != damage())  /* compare current state to prior state */
{
  drive(90,100);    /* robot has been hit, start moving */
  d = damage();     /* get current damage again */
} 
\end{verbatim}

\item[\texttt{void cycle ()}]
        waits until the server has updated the robots' positions.
        This function is not needed except to avoid busy waiting.

\item[\texttt{int speed()}]
        returns the current speed of the robot.
        speed() takes no arguments, and returns the percent of speed,
        0-100.  Note that speed() may not always be the same as the last
        drive(), because of acceleration and deacceleration.

        Example:
\begin{verbatim}
drive(270,100);   /* start drive, due south */
...               /* other instructions */
if (speed() == 0) /* check current speed */
  drive(90,20);   /* ran into south wall or another robot*/
\end{verbatim}

\item[\texttt{int loc\UL x ()} and \texttt{int loc\UL y ()}]
        return the robot's current x (respectively, y) axis location.
        They take no arguments, and returns 0-999.

        Example:
\begin{verbatim}
drive (180,50);  /* start heading for west wall */
while (loc_x() > 20)
  cycle ();      /* do nothing until we are close */
drive (180,0);   /* stop drive */
\end{verbatim}
\end{description}

\section{Organization of the work}

       Your task for this week is to write down \texttt{robots.h} and
       to decide the interfaces between the three modules of the server
       program\footnote{The networking group will write the client
         interface too}: networking, game logic and graphics.  This is
       basically a set of ``struct''s containing info about robots and
       missiles.
       
\section{Network}

      \begin{enumerate}
        \item Send current location to the robots - only if is the start cycle or they're not busy
        \item Move and explode missiles - not the start cycle
        \item Update display - TODO ( Eberli e Dominguez )
        \item Wait 10 ms (minus the time that was used to update the display) -  
        \item Receive commands from the robots - If no command received robot is_busy = true, 
        \item Send command replies to the robots - send FLAG ( 1 if ok 0 if not  ) + MESSAGE if necessary
        \item Move robots
        \end{enumerate}


\end{document}
